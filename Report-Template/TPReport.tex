% This example An LaTeX document showing how to use the l3proj class to
% write your report. Use pdflatex and bibtex to process the file, creating 
% a PDF file as output (there is no need to use dvips when using pdflatex).

% Modified 

\documentclass{l3proj}
\begin{document}
\title{An Example Project}
\author{Michael Cromie -- 1003492\\
        Edvin Malinovskis -- 2039411 \\
        Andrew McDonald -- 1102115  \\
        Fraser Leishman -- 1102103 \\
        Matthew Paterson -- 1102374 \\
        }
\date{\today}
\maketitle
\begin{abstract}

The abstract goes here

\end{abstract}
\educationalconsent
\tableofcontents
%==============================================================================
\chapter{Introduction}
\label{intro}
\section{Project Synopsis}
The school of Computing Science are looking for an exciting demonstration for their 750kg robot to perform. The aim of this project is to design and implement such a demonstration.
\section{Background}
The robot belongs to a much larger project, named CLoPeMa. CLoPeMa is an EU-FP7 project, meaning it is funded by the EU as a recognised science and technological research project. This project runs across four universities in Europe those being: University of Glasgow, University of Genoa(Italy), Czech Tecnical University and the Centre for Research \& Technology Hellas (Greece). 
The main aims of the CLoPeMa project is to use the cameras attached to the robot's arms in order to manipulate, percieve and fold a variety of textiles. This is a really fascinating and challenging project from a mechanics point of view, but doesn't provide very stimulating demonstrations.
\section{Motivation}
// I've no idea what to write here. Something about Siebert having a robot that needs a demo?
\section{Aims}
The aim of this project is to design and implement a sequence of actions for the robot to perform that is not only intelectually stimulating, but also exciting to watch. 
Our overall goal was to come up with a demonstration that would be engaging enough to be shown at future university events in an attempt to appeal to prospective students in years to come.
\section{Outline}
The high level description of the demonstration which we came up with was for the robot to:
\begin{itemize}
\item take a photograph of a person in the room from one of it's in-built cameras
\item pick up a pen from a table in front of it
\item run line detection software on the image
\item draw the image on a sheet of paper also on the table
\end{itemize}
The remainder of the report will explain each stage of the demonstration further and the difficulties we faced at each stage. The report will also describe our alternative design choices. //I don't like that phrase

%==============================================================================
\chapter{Design}
\label{design}

%==============================================================================
\chapter{Implementation}
\label{impl}

%==============================================================================
\chapter{Evaluation}


%==============================================================================
\chapter{Conclusion}

%==============================================================================
\section{Contributions}


%==============================================================================
\bibliographystyle{plain}
\bibliography{example}
\end{document}
