
% Template source - http://www.howtotex.com


% http://infolab.stanford.edu/~widom/paper-writing.html - tips on writing a paper
% https://pages.cs.wisc.edu/~zuyu/files/writing-hints.pdf

\documentclass[paper=a4, fontsize=11pt]{scrartcl}
\usepackage[T1]{fontenc}

\usepackage[english]{babel}															% English language/hyphenation
\usepackage[protrusion=true,expansion=true]{microtype}	
\usepackage{amsmath,amsfonts,amsthm} % Math packages
\usepackage[pdftex]{graphicx}	
\usepackage{url}


%%% Custom sectioning
\usepackage{sectsty}
\allsectionsfont{\centering \normalfont\scshape}


%%% Custom headers/footers (fancyhdr package)
\usepackage{fancyhdr}
\pagestyle{fancyplain}
\fancyhead{}											% No page header
\fancyfoot[L]{}											% Empty 
\fancyfoot[C]{}											% Empty
\fancyfoot[R]{\thepage}									% Pagenumbering
\renewcommand{\headrulewidth}{0pt}			% Remove header underlines
\renewcommand{\footrulewidth}{0pt}				% Remove footer underlines
\setlength{\headheight}{13.6pt}


%%% Equation and float numbering
\numberwithin{equation}{section}		% Equationnumbering: section.eq#
\numberwithin{figure}{section}			% Figurenumbering: section.fig#
\numberwithin{table}{section}				% Tablenumbering: section.tab#


%%% Maketitle metadata
\newcommand{\horrule}[1]{\rule{\linewidth}{#1}} 	% Horizontal rule

\title{
		\usefont{OT1}{bch}{b}{n}
		\normalfont \normalsize \textsc{School of Computing Science} \\ [25pt]
		\horrule{0.5pt} \\[0.4cm]
		\huge Team U CloPeMa Project \\
		\horrule{2pt} \\[0.5cm]
		}

\author{
	\usefont{OT1}{bch}{b}{n}
    	\normalfont Michael Cromie\\
    	\and
    	Fraser Leishman\\
    	\and
    	Edvin Malinovskis\\
    	\and
    	Andrew McDonald\\
    	\and
    	Matthew Paterson
	}
        
\date{
		\usefont{OT1}{bch}{b}{n}
		\normalfont \normalsize \today
        }


%%% Begin document
\begin{document}
\maketitle


\abstract \noindent{ \textit{abstract written at the end and is a condensed overview of whole project}}



\section{Introduction}

The University of Glasgow hosts regular open days, where prospective staff and students come and see what it's like to be a part of the University. During these events, the numerous schools host talks and demonstrations enticing new students into their degree subject.\\
The School of Computing Science at the University of Glasgow require a new interesting demonstration for their large robot, CloPeMa, which is to be shown at such open days and other University events.\\
The main difficulty with the development of such demonstrations lies within the software that the robot relies upon being very tempromental. In order to implement and run new demonstrations on the robot, it takes a multitude of software all communicating together - which seems to have been the downfall on previous attempts.\\
To overcome this we propose to familiarise ourselves with the software first through reading various versions of documentation and through writing numerous "hello world" demonstrations before we start the implementation of our main demonstration.\\









%%% End document
\end{document}
